\documentclass[10pt, presentation]{beamer}
%[10pt,presentation,german,aspectratio=1610,hyperref={pdfpagemode=FullScreen}]
\usetheme{Madrid}
\usecolortheme{default}
\usepackage[backend=bibtex,style=numeric]{biblatex}
\addbibresource{refs.bib}


%\hypersetup{pdfstartview={Fit}}


\title[Machine Learning Approaches for Detection of DDoS Attacks in IoT Networks] %optional
{Machine Learning Approaches for Detection of DDoS Attacks in IoT Networks}

\subtitle{}

\author[Adrian Gruszczynski] % (optional, for multiple authors)
{Adrian Gruszczynski}

\institute[Freie Universität Berlin] % (optional)
{
  Institute of Computer Science\\
  Freie Universität Berlin
}

\date[2022] % (optional)
{Seminar IoT \& Security, July 2022}

\begin{document}

\frame{\titlepage}
%
%\begin{frame}
%\frametitle{Table of Contents}
%
%
%\tableofcontents
%\end{frame}
\section{Introduction}
%---------------------------------------------------------
    \begin{frame}
        \frametitle{Table of Contents}
        \begin{enumerate}
            \item Introduction
            \item Background
            \begin{enumerate}
                \item Botnets
                \item DDoS
                \item Mirai
                \item Machine Learning 101
            \end{enumerate}
            \item Classification of IoT Malware based on Image Recognition
            \item Conclusion
        \end{enumerate}
    \end{frame}
%---------------------------------------------------------

% Introduciton Motivation
%---------------------------------------------------------
\begin{frame}
\frametitle{Introduction}
  \begin{itemize}
    \item Internet of Things is constantly developing
      \begin{itemize}
          \item Estimated 20.4 billion connected devices worldwide in 2022\ \cite{WEBSITE:1}
          \item Applications in various domains e.g.\ smart cities, smart healthcare, autonomous vehicles, industry 4.0, smart grids, etc.
      \end{itemize}
      \item Security and privacy are crucial
      \begin{itemize}
          \item Challenging constrains due to hardware and networking limitations
          \item Heterogeneous networks producing large amounts of data
          \item Low security standards, devices use default credentials
      \end{itemize}
    \item Fertile ground for privacy and security attacks
  \end{itemize}
\end{frame}
%---------------------------------------------------------

% Introduciton DDos
%---------------------------------------------------------
\begin{frame}
\frametitle{Introduction}
  \begin{itemize}
    \item Insecure IoT devices may threaten critical Internet infrastructure
      \begin{itemize}
          \item Using vulnerable consumer IoT devices becomes a common technique for orchestrating DDoS attacks
          \item Mirai botnet disrupted DNS service of Dyn and significantly limited accessibility of popular services such as Github, Netflix and Amazon
      \end{itemize}
      \item Intelligent system monitoring leveraging ML/DL methods provides a solution for threat detection
      \begin{itemize}
          \item Anomaly detection can facilitate detection of malicious traffic
          \item Prediction of future attacks by learning from existing examples
        \end{itemize}
  \end{itemize}
\end{frame}

%---------------------------------------------------------
\section{Background}

% Botnets
%---------------------------------------------------------
\begin{frame}
\frametitle{Botnets}
  \begin{itemize}
    \item Group of devices infected by malware
    \item The size varies from hundreds to hundreds of thousands of devices
    \item Can be controlled remotely by an attacker to execute malicious activities
    \begin{itemize}
          \item Phishing
          \item Spamming
          \item DDoS
      \end{itemize}
    \item IoT provides an ideal foundation for botnets and carrying out DDoS attacks
      \begin{itemize}
          \item Easy target due to low security measures in IoT
          \item High number of hackable devices
          \item Massive pool of legitimate IP addresses and sources of traffic
        \end{itemize}
  \end{itemize}
\end{frame}

% DDoS
%---------------------------------------------------------
\begin{frame}
\frametitle{DDoS}
  \begin{itemize}
    \item Cyberattack that targets a host, network or infrastructure
    \item The goal is to render the target unavailable for others by exhausting its resources (Bandwidth, Memory, CPU, etc.)
    \item Attacks on critical internet infrastructure (DNS) have an enormous impact
    \item Thanks to the broad availability of botnets DDoS is a simple yet powerful weapon
    \item DDoS attacks happen on two levels:
  \begin{itemize}
      % Dont finish 3 way handshake or send a packet with origin IP adress same as receiving to crash the server
      \item Network-level: exploits network layer protocols e.g.\ TCP, UDP, IP, etc.
      % Send POST request very slowly or request
      \item Application-level: exploits application layer protocols e.g.\ HTTP, DNS, etc.
  \end{itemize}
  \item There is a number of DDoS techniques including:
      \begin{itemize}
      % Dont finish 3 way handshake or send a packet with origin IP adress same as receiving to crash the server
      \item Amplification: generate most traffic with least amount of bandwidth
      % Send POST request very slowly or request
      \item Reflection: used in combination with IP-spoofing to hide the origin IP
  \end{itemize}
  \end{itemize}
\end{frame}

%---------------------------------------------------------

% Mirai botnet
%---------------------------------------------------------
\begin{frame}
\frametitle{Mirai}
  \begin{itemize}
      % Embed figure 5 from 7178164.pdf
    \item DDoS capable malware taht first appeared in 2016
    \item Responsible for the biggest scale DDoS attack ever recorded peaking at 1.2Tbps
    \begin{itemize}
        \item Rendered popular internet services unavailable (Github, Netflix, Amazon, etc.)
        \item Impacted United States and Europe
    \end{itemize}
    \item Botnet of approximately ~500k compromised devices
    \item Finish methods, discussion and conclusion, correct spelling, create slides for presentation
    \item Source code was published causing a number of similar attacks to follow
  \end{itemize}
\end{frame}

%---------------------------------------------------------

%---------------------------------------------------------
%Refs
    \begin{frame}
        \frametitle{References}
        \printbibliography
    \end{frame}
%\end{frame}

\end{document}