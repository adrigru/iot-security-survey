\documentclass[10pt, presentation]{beamer}
%[10pt,presentation,german,aspectratio=1610,hyperref={pdfpagemode=FullScreen}]
\usetheme{Madrid}
\usecolortheme{default}
\usepackage[backend=bibtex,style=numeric]{biblatex}
\addbibresource{refs.bib}


%\hypersetup{pdfstartview={Fit}}


\title[Machine Learning Approaches for Detection of DoS Attacks in IoT Networks] %optional
{Machine Learning Approaches for Detection of DoS Attacks in IoT Networks}

\subtitle{}

\author[Adrian Gruszczynski] % (optional, for multiple authors)
{Adrian Gruszczynski}

\institute[Freie Universität Berlin] % (optional)
{
  Institute of Computer Science\\
  Freie Universität Berlin
}

\date[2022] % (optional)
{Seminar IoT \& Security, May 2022}

\begin{document}

\frame{\titlepage}
%
%\begin{frame}
%\frametitle{Table of Contents}
%
%
%\tableofcontents
%\end{frame}
\section{Introduction}
%---------------------------------------------------------
    \begin{frame}
        \frametitle{Table of Contents}
        \begin{enumerate}
            \item Introduction
            \item Tentative report outline
            \item References and related work
            \item Tentative report schedule
        \end{enumerate}
    \end{frame}
%---------------------------------------------------------

% Introduciton Motivation
%---------------------------------------------------------
\begin{frame}
\frametitle{Motivation}
  \begin{itemize}
    \item Internet of Things is constantly developing
      \begin{itemize}
          \item Estimated 20.4 billion connected devices worldwide in 2022\ \cite{WEBSITE:1}
          \item Applications in various domains e.g.\ smart cities, smart healthcare, autonomous vehicles, industry 4.0, smart grids, etc.
      \end{itemize}
      \item Security and privacy are crucial
      \begin{itemize}
          \item Challenging constrains due to hardware and networking limitations
          \item Heterogeneous networks producing large amounts of data
      \end{itemize}
    \item Fertile ground for privacy and security attacks
  \end{itemize}
\end{frame}
%---------------------------------------------------------

% Introduciton DDos
%---------------------------------------------------------
\begin{frame}
\frametitle{Motivation}
  \begin{itemize}
    \item Insecure IoT devices may threaten critical Internet infrastructure
      \begin{itemize}
          \item Using vulnerable consumer IoT devices becomes a common technique for orchestrating DDoS attacks
          \item Mirai botnet disrupted DNS service of Dyn and significantly limited accessibility of popular services such as Github, Netflix and Amazon
      \end{itemize}
      \item Intelligent system monitoring  leveraging ML/DL methods provides a solution for threat detection
      \begin{itemize}
          \item Anomaly detection can facilitate detection of malicious traffic
          \item Prediction of future attacks by learning from existing examples
        \end{itemize}
  \end{itemize}
\end{frame}

%---------------------------------------------------------
\section{Report structure}

% Preliminary report structure
%---------------------------------------------------------
\begin{frame}
\frametitle{Preliminary report structure}
  \begin{itemize}
    \item Introduction
      \begin{itemize}
          \item What is the domain?
          \item What are the challenges?
          \item Why is it important?
      \end{itemize}
      \item Background and related work
      \begin{itemize}
          \item Network anomaly detection
          \item ML/DL for IoT security
          \item DDos attacks in IoT networks
        \end{itemize}
      \item Theoretical framework and architectural design
      \begin{itemize}
          \item System architecture and performance measures
          \item k-nearest neighbours
          \item Boosting
          \item Artificial neural network
        \end{itemize}
      \item Discussion
      \item Conclusion
  \end{itemize}
\end{frame}

\section{Related work}

% Related work
%---------------------------------------------------------
\begin{frame}
\frametitle{Related work}
  \begin{itemize}
    \item A Survey of Machine and Deep Learning Methods for Internet of Things (IoT) Security (IEE Communications Surveys \& Tutorials, cited by 425)
    \item Real-time anomaly detection systems for Denial-of-Service attacks by weighted k-nearest-neighbor classifiers (Expert Systems with Applications, cited by 151)
    \item Machine Learning DDoS Detection for Consumer Internet of Things Devices (2018 IEEE Security and Privacy Workshops (SPW), cited by 425)
    \item Boosting-Based DDoS Detection in Internet of Things Systems (IEEE Internet of Things Journal 2022, cited by 20)
    \item Detection of known and unknown DDoS attacks using Artificial Neural Networks (Neurocomputing, cited by 307)
  \end{itemize}
\end{frame}

%---------------------------------------------------------

\section{Report schedule}
% Report schedule
%---------------------------------------------------------
\begin{frame}
\frametitle{Report schedule}
  \begin{itemize}
    \item 10th May 2022: Presentation
    \item Report outline, finish draft introduction and related work, start working on methods
    \item 8th June 2022: Preliminary version of the report
    \item Finish methods, discussion and conclusion, correct spelling, create slides for presentation
    \item 3rd July 2022: Deadline for report submission
    \item 6th July 2022: Final presentation
  \end{itemize}
\end{frame}

%---------------------------------------------------------

%---------------------------------------------------------
%Refs
    \begin{frame}
        \frametitle{References}
        \printbibliography
    \end{frame}
%\end{frame}

\end{document}